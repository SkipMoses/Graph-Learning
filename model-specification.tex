% Options for packages loaded elsewhere
\PassOptionsToPackage{unicode}{hyperref}
\PassOptionsToPackage{hyphens}{url}
%
\documentclass[
]{article}
\usepackage{amsmath,amssymb}
\usepackage{lmodern}
\usepackage{ifxetex,ifluatex}
\ifnum 0\ifxetex 1\fi\ifluatex 1\fi=0 % if pdftex
  \usepackage[T1]{fontenc}
  \usepackage[utf8]{inputenc}
  \usepackage{textcomp} % provide euro and other symbols
\else % if luatex or xetex
  \usepackage{unicode-math}
  \defaultfontfeatures{Scale=MatchLowercase}
  \defaultfontfeatures[\rmfamily]{Ligatures=TeX,Scale=1}
\fi
% Use upquote if available, for straight quotes in verbatim environments
\IfFileExists{upquote.sty}{\usepackage{upquote}}{}
\IfFileExists{microtype.sty}{% use microtype if available
  \usepackage[]{microtype}
  \UseMicrotypeSet[protrusion]{basicmath} % disable protrusion for tt fonts
}{}
\makeatletter
\@ifundefined{KOMAClassName}{% if non-KOMA class
  \IfFileExists{parskip.sty}{%
    \usepackage{parskip}
  }{% else
    \setlength{\parindent}{0pt}
    \setlength{\parskip}{6pt plus 2pt minus 1pt}}
}{% if KOMA class
  \KOMAoptions{parskip=half}}
\makeatother
\usepackage{xcolor}
\IfFileExists{xurl.sty}{\usepackage{xurl}}{} % add URL line breaks if available
\IfFileExists{bookmark.sty}{\usepackage{bookmark}}{\usepackage{hyperref}}
\hypersetup{
  pdftitle={Comments on binomial graph learning},
  hidelinks,
  pdfcreator={LaTeX via pandoc}}
\urlstyle{same} % disable monospaced font for URLs
\usepackage[margin=1in]{geometry}
\usepackage{graphicx}
\makeatletter
\def\maxwidth{\ifdim\Gin@nat@width>\linewidth\linewidth\else\Gin@nat@width\fi}
\def\maxheight{\ifdim\Gin@nat@height>\textheight\textheight\else\Gin@nat@height\fi}
\makeatother
% Scale images if necessary, so that they will not overflow the page
% margins by default, and it is still possible to overwrite the defaults
% using explicit options in \includegraphics[width, height, ...]{}
\setkeys{Gin}{width=\maxwidth,height=\maxheight,keepaspectratio}
% Set default figure placement to htbp
\makeatletter
\def\fps@figure{htbp}
\makeatother
\setlength{\emergencystretch}{3em} % prevent overfull lines
\providecommand{\tightlist}{%
  \setlength{\itemsep}{0pt}\setlength{\parskip}{0pt}}
\setcounter{secnumdepth}{-\maxdimen} % remove section numbering
\ifluatex
  \usepackage{selnolig}  % disable illegal ligatures
\fi

\title{Comments on binomial graph learning}
\author{}
\date{\vspace{-2.5em}}

\begin{document}
\maketitle

\hypertarget{binary-graph-learning-model}{%
\subsection{Binary graph learning
model}\label{binary-graph-learning-model}}

\hypertarget{version-1-eigenvector-matrix-from-graph-laplacian}{%
\subsubsection{Version 1 (Eigenvector matrix from Graph
Laplacian)}\label{version-1-eigenvector-matrix-from-graph-laplacian}}

Let \(Y_{i,j}\) denote the measurement on the node \(i\) at round \(j\),
where \(j = 1, \dots, M\), and \(i = 1, \dots, N\). \(Y_{i,j}\) is a
binomial signal that can be 1, or 0. Suppose the signals at round \(j\)
denoted by \(Y[, j]\) for all \(N\) nodes are independent of the signals
at round \(k\) denoted by \(Y[,k ]\), for \(i \neq k\). Let \(p_{i,j}\)
denote the probability of \(Y_{i,j} = 1\). Our model assumes
\[\text{logit}(p_{i,j}) = \alpha_j + (\chi h )_i,\]

where \(\chi\) is the eigenvector matrix from Graph Laplacian \(L\),
\(h\) is a vector of latent factors that governs \(p_{i, j}\) through
\(\chi\), and \(\alpha_j\) is a round specific parameter at round \(j\).

\hypertarget{version-2-adjacency-matrix-from-graph}{%
\subsubsection{Version 2 (Adjacency matrix from
Graph)}\label{version-2-adjacency-matrix-from-graph}}

\hypertarget{graph}{%
\paragraph{Graph}\label{graph}}

We consider a weighted undirected graph \(G = (V, E)\), with the
vertices set \(V = {1, 2, \dots, N}\), and edge set \(E\). Let
\(\mathbf{A}\) denote the weighted adjacency matrix for the graph \(G\).
In the case of weighted undirected graph, \(\mathbf{A}\) is a square and
symmetric matrix.

\hypertarget{signals-on-the-graph}{%
\paragraph{Signals on the graph}\label{signals-on-the-graph}}

Let \(Y_{i,j}\) denote the signal on the node \(i\) of graph \(G\) at
round \(j\), where \(j = 1, \dots, M\), and \(i = 1, \dots, N\). We
assume that \(Y_{i,j}\) is a binary signal that can be 1, or 0. Suppose
the signals at round \(j\) denoted by \(Y[, j]\) for all \(N\) nodes are
independent of the signals at round (or stratum) \(k\) denoted by
\(Y[,k ]\), for \(j \neq k\), borrowing the idea of conditional logistic
regression. Let \(p_{i,j}\) denote the probability of \(Y_{i,j} = 1\).
Our model assumes

\begin{equation}
\label{eq:binaryglm}
\text{logit}(p_{i,j}) = \alpha_j + (\mathbf{A} h )_i,
\end{equation}

where \(\mathbf{A}\) is the adjacency matrix from the graph \(G\), \(h\)
is a vector of latent factors that governs \(p_{i, j}\) through
\(\mathbf{A}\) and assumed to be a standard normal random vector, and
\(\alpha_j\) is a round specific parameter at round or stratum \(j\),
and assumed to be normally distributed with mean of 0, and unknown
variance \(\sigma^2\)

\hypertarget{method-of-estimation}{%
\subsection{Method of Estimation}\label{method-of-estimation}}

\hypertarget{conditional-likelihood-for-one-stratum}{%
\subsubsection{Conditional likelihood for one
stratum}\label{conditional-likelihood-for-one-stratum}}

Let \(O_j\) denote the set of nodes at stratum \(j\) that have observed
signals of 1, and let \(Z_j\) denote the set of nodes at stratum \(j\)
that have observed signals of 0. Suppose the number of nodes that have
signals of 1 at stratum \(j\) is \(k_j\). The conditional likelihood
function based on model (\ref{eq:binaryglm}) for stratum \(j\) of size
\(N\), is

\begin{equation}
    \label{eq:stratumlikelihood}
    \begin{aligned}
    &   P(Y_{lj} = 1 \text{ for } l \in O_j, Y_{mj} = 0 \text{ for } m \in Z_j | \sum_{i=1}^N Y_{ij} = k_j) \\   
    & =  {\text{exp} (\sum_{l \in O_j}\mathbf{A}[l, ] h ) \over \sum_{J \in C_{k_j}^N} \text{exp} (\sum_{a \in J}\mathbf{A}[a, ] h ) },
    \end{aligned}
\end{equation}

where \(C_{k_j}^N\) is the set of all subsets of size \(k_j\) of the set
\({1, 2, \dots, N}\).

\hypertarget{conditional-likelihoood-function}{%
\subsubsection{Conditional likelihoood
function}\label{conditional-likelihoood-function}}

The conditional likelihood function for all strata is written as
\begin{equation}
    \label{eq:conlikelihood}
    \begin{aligned}
    &   L(\mathbf{A}, h)  = \prod_{j=1}^M {\text{exp} (\sum_{l \in O_j}\mathbf{A}[l, ] h ) \over \sum_{J \in C_{k_j}^N} \text{exp} (\sum_{a \in J}\mathbf{A}[a, ] h ) },
    \end{aligned}
\end{equation}

\hypertarget{conditional-log-likelihood-function}{%
\subsubsection{Conditional log likelihood
function}\label{conditional-log-likelihood-function}}

The full conditional log likelihood function is the sum of the log
likelihoods for each stratum and can be written as

\begin{equation}
    \label{eq:condloglikelihood}
    \begin{aligned}
    &   \text{log}(L(\mathbf{A}, h))  = \sum_{j=1}^M \sum_{l \in O_j}\mathbf{A}[l, ] h  - \sum_{j=1}^M \text{log}\sum_{J \in C_{k_j}^N} \text{exp} (\sum_{a \in J}\mathbf{A}[a, ] h ) 
    & = \sum_{j=1}^M \mathbf{A} h y_j  - \sum_{j=1}^M \text{log}\sum_{J \in C_{k_j}^N} \text{exp} (\sum_{a \in J}\mathbf{A}[a, ] h )
    \end{aligned}
\end{equation}

\hypertarget{the-estimator}{%
\subsubsection{The Estimator}\label{the-estimator}}

The maximum likelihood estimator is then defined as \(\mathbf{A}\) which
maximize (\ref{eq:condloglikelihood}), which is equivalent to minimizing
the negative conditional log likelihood.

\hypertarget{miscellaneous}{%
\subsection{Miscellaneous}\label{miscellaneous}}

Constraints:

\begin{itemize}
\tightlist
\item
  Case \(\chi\): Here I imagine the constraints are on the Laplacian
\end{itemize}

\begin{equation}
\begin{aligned}
&\quad \text{tr}(L) = N, \\
&\quad L_{i,j} = L_{j,i} \leq 0, \phantom{..} i \neq j, \\
&\quad L\cdot \textbf{1} = \textbf{0} \\
\end{aligned}
\end{equation}

\begin{itemize}
\tightlist
\item
  Case with adjacency matrix:
\end{itemize}

\begin{equation}
\begin{aligned}
&\quad A_{i,j} = 0 \text{ if } i = j, \\
&\quad A_{i,j} = A_{j,i} \geq 0, \phantom{..} \text{ if } i \neq j
\end{aligned}
\end{equation}

\begin{itemize}
\tightlist
\item
  Consider the quadratic form \[
  x^TLx = \frac{1}{2}\sum_{(i,j)\in E}w_{i,j}(f(i) - f(j))^2
  \]
\end{itemize}

Because \(x\) is a binomial, this expression sums the weights of edeges
with incident nodes that have differing signal values. Suppose the graph
has a constant signal, then \(x^TLx = 0\). Suppose, we are in the worst
case when incident nodes for each edge have a different signal; in this
case \(x^TLx = \frac{1}{2}\sum_{(i,j) \in E}w_{i,j}\). Note, we can
bound \(x^TLx\) from below by \(\min{w_{i,j}}\) in the case were a
single edge has incident nodes with differing signals. We can use these
bounds to gauge the quality of model.

Sigmoid computation, Bound and Branch method of optimization

hint for factor analysis solution might be useful because there is
another object function varimax.

\url{https://conservancy.umn.edu/bitstream/handle/11299/95957/Choi_umn_0130E_11451.pdf?sequence=1\&isAllowed=y}

\url{https://web.stanford.edu/~boyd/papers/pdf/max_sum_sigmoids.pdf}

\begin{itemize}
\item
  complete separation do not exist. observable metric ?, it is a
  potential new topic to be able to identify whether complete separation
  exists. possible discuss it in the Discussion and future work.
\item
  While working on this one, you may also consider implementing our
  regressor paper.
\end{itemize}

\hypertarget{model-specification}{%
\subsection{Model Specification}\label{model-specification}}

\hypertarget{our-goal-is-to-estimate-the-graph-laplacian}{%
\subsection{Our goal is to estimate the Graph
Laplacian}\label{our-goal-is-to-estimate-the-graph-laplacian}}

\begin{itemize}
\tightlist
\item
  We do not have to use the eigenvector matrix \(\chi\). Instead, we may
  consider adjacency matrix \(A\) in place of \(\chi\). This is a viable
  direction.
\end{itemize}

Maximum Likelihood estimation, Quasi likelihood estimation,

\end{document}
